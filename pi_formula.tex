\documentclass[varwidth,convert={density=300,size=800x800,outext=.png}]{standalone}

%/*
% * This program source code file is part of KiCad, a free EDA CAD application.
% *
% * Copyright (C) 2015 Jonas Stein news@jonasstein.de
% * Copyright (C) 2015 KiCad Developers, see CHANGELOG.TXT for contributors.
% *
% * This program is free software; you can redistribute it and/or
% * modify it under the terms of the GNU General Public License
% * as published by the Free Software Foundation; either version 3
% * of the License, or (at your option) any later version.
% *
% * This program is distributed in the hope that it will be useful,
% * but WITHOUT ANY WARRANTY; without even the implied warranty of
% * MERCHANTABILITY or FITNESS FOR A PARTICULAR PURPOSE.  See the
% * GNU General Public License for more details.
% *
% * You should have received a copy of the GNU General Public License
% * along with this program; if not, you may find one here:
% * http://www.gnu.org/licenses/old-licenses/gpl-2.0.html
% * or you may search the http://www.gnu.org website for the version 2 license,
% * or you may write to the Free Software Foundation, Inc.,
% * 51 Franklin Street, Fifth Floor, Boston, MA  02110-1301, USA
% */
%
% create .png file with
% pdflatex -shell-escape pi_formula.tex


\usepackage{amssymb}
\usepackage{tikz}
\usepackage{siunitx}
\usepackage{enumitem}
\usepackage{mathtools}

\newcommand{\Zin}{Z_\mathrm{in}}
\newcommand{\Zout}{Z_\mathrm{out}}

\begin{document}
\begin{description}[labelindent=0pt,labelsep=10pt]
\item[$\Zin$] desired input impedance in \si{\ohm}
\item[$\Zout$] desired output impedance in \si{\ohm}
\item[$a$] attenuation of the RF power or amplitude in \si{\decibel}
\item[$L$] loss $L = 10^{\frac{a}{10}}$ 
\item[$A$] $= (L+1)/(L-1)$
\end{description}

\paragraph{Pi attenuator}
\[R_2 = \frac{L - 1}{2} \cdot \sqrt{\frac{\Zin \cdot \Zout}{L}} \]
\[R_1 = \frac{1}{ \frac{A}{\Zin} -  \frac{1}{R_2}}\]
\[R_3 = \frac{1}{ \frac{A}{\Zout} -  \frac{1}{R_2}}\]

%\[R_1 = \frac{1}{\frac{G+1}{\Zin \cdot (G-1)} - \frac{1}{R_2}}\]
%\[R_2 = \frac{G-1}{2}  \cdot \sqrt{\frac{\Zin \cdot \Zout}{G}}\]
%\[R_3 = \frac{1}{\frac{G+1}{\Zout \cdot (G-1)} - \frac{1}{R_2}}\]

\end{document}